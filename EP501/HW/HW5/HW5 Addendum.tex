\documentclass[12pt,letterpaper]{article}
\usepackage[utf8]{inputenc}
\usepackage{amsmath}
\usepackage{amsfonts}
\usepackage{amssymb}
\usepackage{graphicx}
\usepackage[left=0.75in,right=0.75in,top=0.75in,bottom=0.75in]{geometry}
\author{Paul Yuska}
\title{EP501 HW5 Equations}
\begin{document}
\maketitle

\section*{Problem 1 Equations}
Following the method outlined in Section 8.4.1 of the textbook, the general form of the equation we are working with is:
\begin{equation}
\bar{y}'' + P(x)\bar{y}' + Q(x)\bar{y} = F(x)\;,
\end{equation}
where
\begin{eqnarray*}
P(x) &=& \frac{1}{\epsilon}\frac{d\epsilon}{dx}\\
Q(x) &=& 0\\
F(x) &=& 0\\
y &=& \Phi
\end{eqnarray*}
Substituting second-order centered difference expressions for the derivatives as given by Equations 8.43 and 8.44 in the textbook and rearranging gives a version of Equation 8.46:
\begin{equation}
\label{FDE_P}
\left(1 - \frac{\Delta x}{2}P_i	\right)\Phi_{i-1} - 2\Phi_i + \left(1 + \frac{\Delta x}{2}P_i	\right)\Phi_{i+1} = 0
\end{equation}
$P_i$ can be described by a second-order FDA at the interior points and a first-order FDA at the boundaries. Assuming that the computational grid begins at index 1 at the left boundary and ends at index N at the right boundary,
\begin{eqnarray*}
P(x) &=& \frac{\epsilon_{i+1} - \epsilon_{i-1}}{2\epsilon_i \Delta x}\qquad \textnormal{(interior)}\\
P(x) &=& \frac{\epsilon_{2} - \epsilon_{1}}{\epsilon_1 \Delta x}\qquad \textnormal{(left boundary)}\\
P(x) &=& \frac{\epsilon_{N} - \epsilon_{N-1}}{\epsilon_N \Delta x}\qquad \textnormal{(right boundary)}
\end{eqnarray*}
Substituting the interior $P(x)$ function into (\ref{FDE_P}) gives the FDE for this system:
\begin{equation}
\left(1 - \frac{\epsilon_{i+1} - \epsilon_{i-1}}{4\epsilon_i}\right)\Phi_{i-1} - 2\Phi_i + \left(1 + \frac{\epsilon_{i+1} - \epsilon_{i-1}}{4\epsilon_i}	\right)\Phi_{i+1} = 0
\end{equation}
%Mixed boundary conditions are of the form of Equation 8.93 in the textbook, repeated here:
%\begin{equation}
%\label{mixedBC}
%A\bar{\Phi}(x_B) + B\bar{\Phi}'(x_B) = C
%\end{equation}
%Using second-order approximations as before, (\ref{mixedBC}) can be transformed into
%\begin{equation}
%\Phi_{I+1} = \Phi_{I-1} + \frac{2 \Delta x}{B}(C-Ay_I)\;,
%\end{equation}
%where $I$ represents an index on the boundary. 
Given our two initial conditions and the same index numbering scheme, the two resulting FDEs for the boundaries are
\begin{eqnarray}
1000 &=& \frac{\Phi_{2} - \Phi_{1}}{\Delta x} \qquad \textnormal{(for $x = -a$)}\\
100 &=& \Phi_N \qquad \textnormal{(for $x = a$)}
\end{eqnarray}
The resulting system of equations, expressed in matrix format, is
\begin{gather}
\begin{bmatrix}
-1 & 1 &   & &  & & &  0 \quad\\
\alpha & -2 & \beta &  & & & & \\
 & \alpha & -2 & \beta  & &  & & \\
 \\
 & & & & \ddots & & & \\
 \\
& & & & & \alpha& -2& \beta \quad\\
0 & & & & & & & 1 \quad
\end{bmatrix}
\begin{bmatrix}
\Phi_1\\
\Phi_2\\
\Phi_3\\
\\
\vdots \\
\\
\Phi_{N-1}\\
\Phi_{N}
\end{bmatrix}
=
\begin{bmatrix}
1000 \Delta x\\
0\\
0\\
\\
\vdots \\
\\
0\\
100
\end{bmatrix}
\end{gather}
where blank elements are zero, and $\alpha$ and $\beta$ are given by
\begin{eqnarray*}
\alpha &=& 1 - \frac{\epsilon_{i+1} - \epsilon_{i-1}}{4\epsilon_i}\\
\beta &=& 1 + \frac{\epsilon_{i+1} - \epsilon_{i-1}}{4\epsilon_i}
\end{eqnarray*}
Since the dielectric function varies rapidly at the edges, the derivative of $\Phi$ can be better approximated by a second-order FDA at $x = -a$:
\begin{equation}
\frac{d \Phi(-a)}{dx} = \frac{-\Phi_{3} + 4\Phi_{2} - 3\Phi_1}{2\Delta x}
\end{equation}
\end{document}